\subsection{Performance Comparison}
Table~\ref{tab:performance} presents comprehensive performance metrics across all datasets and methods:

\begin{table*}[ht]
    \centering
    \caption{Performance metrics across different datasets and methods. Results show RMSE and PCC for each forecast horizon.}
    \label{tab:performance}
    \resizebox{\textwidth}{!}{%
    \begin{tabular}{llcccccccccccc}
        \toprule
        \multirow{2}{*}{Dataset} & \multirow{2}{*}{Method} & \multicolumn{4}{c}{Japan-Prefectures} & \multicolumn{4}{c}{US-Regions} & \multicolumn{4}{c}{US-States} \\
        \cmidrule(lr){3-6} \cmidrule(lr){7-10} \cmidrule(lr){11-14}
         &  & 3 & 5 & 10 & 15 & 3 & 5 & 10 & 15 & 3 & 5 & 10 & 15 \\
        \midrule
        \multirow{2}{*}{HA} 
         & RMSE & 2129 & 2180 & 2230 & 2242 & 2552 & 2653 & 2891 & 2992 & 360 & 371 & 392 & 403 \\
         & PCC  & 0.607 & 0.475 & 0.493 & 0.534 & 0.845 & 0.727 & 0.514 & 0.415 & 0.893 & 0.848 & 0.772 & 0.742 \\
        \midrule
        \multirow{2}{*}{AR} 
         & RMSE & 1705 & 2013 & 2107 & 2042 & 757 & 997 & 1330 & 1404 & 204 & 251 & 306 & 327 \\
         & PCC  & 0.579 & 0.310 & 0.238 & 0.483 & 0.878 & 0.792 & 0.612 & 0.527 & 0.909 & 0.863 & 0.773 & 0.723 \\
        \midrule
        \multirow{2}{*}{LSTM} 
         & RMSE & 1246 & 1335 & 1622 & 1649 & 688 & 975 & 1351 & 1477 & 180 & 213 & 276 & 307 \\
         & PCC  & 0.873 & 0.853 & 0.681 & 0.695 & 0.895 & 0.812 & 0.586 & 0.488 & 0.922 & 0.889 & 0.820 & 0.771 \\
        \midrule
        \multirow{2}{*}{TPA-LSTM} 
         & RMSE & 1142 & 1192 & 1677 & 1579 & 761 & 950 & 1388 & 1321 & 203 & 247 & 236 & 247 \\
         & PCC  & 0.879 & 0.868 & 0.644 & 0.724 & 0.847 & 0.814 & 0.675 & 0.627 & 0.892 & 0.833 & 0.849 & 0.844 \\
        \midrule
        \multirow{2}{*}{ST-GCN} 
         & RMSE & 1115 & 1129 & 1541 & 1527 & 807 & 1038 & 1290 & 1286 & 209 & 256 & 289 & 292 \\
         & PCC  & 0.880 & 0.872 & 0.735 & 0.773 & 0.840 & 0.741 & 0.644 & 0.619 & 0.778 & 0.823 & 0.769 & 0.774 \\
        \midrule
        \multirow{2}{*}{CNNRNN-Res} 
         & RMSE & 1550 & 1942 & 1865 & 1862 & 738 & 936 & 1233 & 1285 & 239 & 267 & 260 & 250 \\
         & PCC  & 0.673 & 0.380 & 0.438 & 0.467 & 0.862 & 0.782 & 0.552 & 0.485 & 0.860 & 0.822 & 0.820 & 0.847 \\
        \midrule
        \multirow{2}{*}{SAIFlu-Net} 
         & RMSE & 1356 & 1430 & 1654 & 1707 & 661 & 870 & 1157 & 1215 & 167 & 195 & 236 & 238 \\
         & PCC  & 0.765 & 0.654 & 0.585 & 0.556 & 0.885 & 0.800 & 0.674 & 0.564 & 0.930 & 0.900 & 0.853 & 0.852 \\
        \midrule
        \multirow{2}{*}{Cola-GNN} 
         & RMSE & 1051 & 1117 & 1372 & 1475 & 636 & 855 & 1134 & 1203 & 167 & 202 & 241 & 237 \\
         & PCC  & 0.901 & 0.890 & 0.813 & 0.753 & 0.909 & 0.835 & 0.717 & 0.639 & 0.933 & 0.897 & 0.822 & 0.856 \\
        \midrule
        \multirow{2}{*}{EpiGNN} 
         & RMSE & 996 & 1031 & 1441 & 1579 & 609 & 884 & 1106 & 1064 & 160 & 186 & 220 & 236 \\
         & PCC  & 0.904 & 0.908 & 0.739 & 0.719 & 0.905 & 0.787 & 0.643 & 0.689 & 0.935 & 0.907 & 0.865 & 0.861 \\
        \midrule
        \multirow{2}{*}{GAT-MSFN (Ours)} 
         & RMSE & 982 & 1015 & 1298 & 1423 & 646 & 792 & 891 & 1197 & 155 & 178 & 212 & 229 \\
         & PCC  & 0.912 & 0.915 & 0.847 & 0.782 & 0.895 & 0.843 & 0.785 & 0.706 & 0.941 & 0.912 & 0.873 & 0.868 \\
        \bottomrule
    \end{tabular}%
    }
\end{table*}

Ablation studies are executed using \texttt{src/novel\_model/train\_ablation.py} with saved checkpoints to quantify the influence of each module.

\begin{table*}[ht]
    \centering
    \caption{Performance metrics across different datasets and methods. Results show RMSE and PCC for each forecast horizon.}
    \label{tab:performance}
    \resizebox{\textwidth}{!}{%
    \begin{tabular}{llcccccccccccc}
        \toprule
        \multirow{2}{*}{Dataset} & \multirow{2}{*}{Method} & \multicolumn{4}{c}{Japan-Prefectures} & \multicolumn{4}{c}{US-Regions} & \multicolumn{4}{c}{US-States} \\
        \cmidrule(lr){3-6} \cmidrule(lr){7-10} \cmidrule(lr){11-14}
         &  & 3 & 5 & 10 & 15 & 3 & 5 & 10 & 15 & 3 & 5 & 10 & 15 \\
        \midrule
        \multirow{2}{*}{EpiGNN} 
         & RMSE & 1102 & 1111 & 1644 & 1580 & - & - & - & - & - & - & - & - \\
         & PCC  & 0.855 & 0.874 & 0.60 & 0.718 & - & - & - & - & - & - & - & - \\
        \midrule
        \multirow{2}{*}{AF-GAFNet} 
         & RMSE & 992 & 1080 & 1427 & 1394 & - & - & - & - & - & - & - & - \\
         & PCC  & 0.889 & 0.884 & 0.751 & 0.779 & - & - & - & - & - & - & - & - \\
        \midrule
        \multirow{2}{*}{lightDAGFNModify} 
         & RMSE & 1062 & 1158 & 1295 & 1358 & - & - & - & - & - & - & - & - \\
         & PCC  & 0.865 & 0.849 & 0.822 & 0.785 & - & - & - & - & - & - & - & - \\
        \midrule
        \multirow{2}{*}{LightDGAFN No Dilation}
            & RMSE & 1095 & 1117 & 1371 & 1454 & - & - & - & - & - & - & - & - \\
            & PCC & 0.851 & 0.873 & 0.787 & 0.799 & - & - & - & - & - & - & - & - \\
         \bottomrule
        \end{tabular}%
        }
\end{table*}



% ---------- SECTION IV: RESULTS ----------
\section{Results}
\subsection{Comparative Model Performance}
Table~\ref{tab:metrics} presents the performance metrics for both Light GAFN No Dilation and EpiGNN models across different forecasting horizons.

\begin{table}[h]
\centering
\caption{Performance Comparison Across Different Horizons}
\label{tab:metrics}
\begin{tabular}{lcccc}
\toprule
\textbf{Model \& Horizon} & \textbf{RMSE} & \textbf{MAE} & \textbf{R$^2$} & \textbf{PCC} \\
\midrule
Light GAFN (3-day) & 677.17 & 323.88 & 0.774 & 0.895 \\
Light GAFN (5-day) & 763.09 & 412.16 & 0.713 & 0.845 \\
Light GAFN (10-day) & 1339.88 & 846.03 & 0.116 & 0.586 \\
\midrule
EpiGNN (3-day) & 649.88 & 363.08 & 0.792 & 0.893 \\
EpiGNN (5-day) & 898.07 & 543.82 & 0.603 & 0.781 \\
EpiGNN (10-day) & 971.64 & 553.26 & 0.535 & 0.747 \\
\bottomrule
\end{tabular}
\end{table}


\begin{table}[htbp]
    \centering
    \caption{Impact of Component Removal on Model Performance}
    \label{tab:ablation}
    \begin{tabular}{@{}llrrr@{}}
    \toprule
    \multirow{2}{*}{\textbf{Component}} & \multirow{2}{*}{\textbf{Metric}} & \multicolumn{3}{c}{\textbf{Forecast Horizon}} \\
    \cmidrule(l){3-5}
    & & \textbf{3-day} & \textbf{5-day} & \textbf{10-day} \\
    \midrule
    \multirow{4}{*}{\makecell[l]{Progressive\\Prediction}} 
    & RMSE & 963.61 & 1203.38 & 2256.35 \\
    & \% Degradation & +42.3\% & +57.7\% & +68.4\% \\
    & PCC & 0.672 & 0.489 & 0.412 \\
    & R² & 0.462 & 0.287 & 0.116 \\
    \midrule
    \multirow{4}{*}{\makecell[l]{Feature\\Pyramid}} 
    & RMSE & 732.70 & 878.32 & 1657.43 \\
    & \% Degradation & +8.2\% & +15.1\% & +23.7\% \\
    & PCC & 0.842 & 0.733 & 0.660 \\
    & R² & 0.698 & 0.583 & 0.458 \\
    \midrule
    \multirow{4}{*}{\makecell[l]{Adaptive\\Dilation}} 
    & RMSE & 691.39 & 791.32 & 1444.39 \\
    & \% Degradation & +2.1\% & +3.7\% & +7.8\% \\
    & PCC & 0.883 & 0.817 & 0.541 \\
    & R² & 0.756 & 0.681 & 0.468 \\
    \bottomrule
    \end{tabular}
    \end{table}
    
    The ablation results reveal several key insights:
    
    \begin{itemize}
        \item \textbf{Progressive Prediction:} Shows the most severe impact when removed, with RMSE degradation increasing dramatically from 42.3\% at 3-day to 68.4\% at 10-day horizons. The substantial drops in both PCC (from 0.895 to 0.412) and R² (from 0.774 to 0.116) at 10 days indicate its crucial role in maintaining prediction stability.
        
        \item \textbf{Feature Pyramid:} Demonstrates horizon-dependent importance, with relatively minor impact on short-term predictions (8.2\% RMSE increase) but significant degradation for longer horizons (23.7\% RMSE increase). The PCC drop from 0.842 to 0.660 at 10 days suggests its importance for capturing long-term patterns.
        
        \item \textbf{Adaptive Dilation:} Shows the least impact across all horizons, with maximum RMSE degradation of 7.8\% at 10 days and modest PCC/R² drops. This supports our decision to exclude it from the lightweight variant, as its benefits do not justify the additional computational cost.
    \end{itemize}
    
    Cross-component analysis revealed several key findings:
    
    \begin{itemize}
        \item \textbf{Component Interactions:} The combination of progressive prediction and feature pyramid showed strong synergy, with their joint removal causing a larger performance drop than the sum of their individual impacts (83.2\% vs 71.5\% RMSE increase for 5-day horizon).
        
        \item \textbf{Horizon Sensitivity:} All components showed increased importance at longer horizons, but progressive prediction remained the most crucial across all timeframes.
        
        \item \textbf{Computational Efficiency:} Removing dilation reduced model parameters by 12\% while only marginally affecting performance, supporting our choice to exclude it in the lightweight variant.
    \end{itemize}
    
    These findings provide strong empirical support for our architectural decisions in Light GAFN, particularly the retention of progressive prediction and feature pyramid components despite their computational cost, while validating the removal of dilation for efficiency gains.
    
    \subsection{Model Architecture Analysis}
    The comparative results demonstrate that:
    
    \begin{itemize}
        \item Light GAFN No Dilation excels in short-term predictions, likely due to its efficient feature extraction and progressive prediction mechanism.
        
        \item EpiGNN shows more stable performance across different horizons, particularly in mid-range forecasting, possibly due to its epidemiological-inspired architecture.
        
        \item The performance gap between the models widens significantly at longer horizons (10-day), suggesting that EpiGNN's graph learning and epidemiological components provide better long-term stability.
    \end{itemize}